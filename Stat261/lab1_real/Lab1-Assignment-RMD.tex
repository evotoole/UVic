% Options for packages loaded elsewhere
\PassOptionsToPackage{unicode}{hyperref}
\PassOptionsToPackage{hyphens}{url}
%
\documentclass[
]{article}
\usepackage{amsmath,amssymb}
\usepackage{iftex}
\ifPDFTeX
  \usepackage[T1]{fontenc}
  \usepackage[utf8]{inputenc}
  \usepackage{textcomp} % provide euro and other symbols
\else % if luatex or xetex
  \usepackage{unicode-math} % this also loads fontspec
  \defaultfontfeatures{Scale=MatchLowercase}
  \defaultfontfeatures[\rmfamily]{Ligatures=TeX,Scale=1}
\fi
\usepackage{lmodern}
\ifPDFTeX\else
  % xetex/luatex font selection
\fi
% Use upquote if available, for straight quotes in verbatim environments
\IfFileExists{upquote.sty}{\usepackage{upquote}}{}
\IfFileExists{microtype.sty}{% use microtype if available
  \usepackage[]{microtype}
  \UseMicrotypeSet[protrusion]{basicmath} % disable protrusion for tt fonts
}{}
\makeatletter
\@ifundefined{KOMAClassName}{% if non-KOMA class
  \IfFileExists{parskip.sty}{%
    \usepackage{parskip}
  }{% else
    \setlength{\parindent}{0pt}
    \setlength{\parskip}{6pt plus 2pt minus 1pt}}
}{% if KOMA class
  \KOMAoptions{parskip=half}}
\makeatother
\usepackage{xcolor}
\usepackage[margin=1in]{geometry}
\usepackage{color}
\usepackage{fancyvrb}
\newcommand{\VerbBar}{|}
\newcommand{\VERB}{\Verb[commandchars=\\\{\}]}
\DefineVerbatimEnvironment{Highlighting}{Verbatim}{commandchars=\\\{\}}
% Add ',fontsize=\small' for more characters per line
\usepackage{framed}
\definecolor{shadecolor}{RGB}{248,248,248}
\newenvironment{Shaded}{\begin{snugshade}}{\end{snugshade}}
\newcommand{\AlertTok}[1]{\textcolor[rgb]{0.94,0.16,0.16}{#1}}
\newcommand{\AnnotationTok}[1]{\textcolor[rgb]{0.56,0.35,0.01}{\textbf{\textit{#1}}}}
\newcommand{\AttributeTok}[1]{\textcolor[rgb]{0.13,0.29,0.53}{#1}}
\newcommand{\BaseNTok}[1]{\textcolor[rgb]{0.00,0.00,0.81}{#1}}
\newcommand{\BuiltInTok}[1]{#1}
\newcommand{\CharTok}[1]{\textcolor[rgb]{0.31,0.60,0.02}{#1}}
\newcommand{\CommentTok}[1]{\textcolor[rgb]{0.56,0.35,0.01}{\textit{#1}}}
\newcommand{\CommentVarTok}[1]{\textcolor[rgb]{0.56,0.35,0.01}{\textbf{\textit{#1}}}}
\newcommand{\ConstantTok}[1]{\textcolor[rgb]{0.56,0.35,0.01}{#1}}
\newcommand{\ControlFlowTok}[1]{\textcolor[rgb]{0.13,0.29,0.53}{\textbf{#1}}}
\newcommand{\DataTypeTok}[1]{\textcolor[rgb]{0.13,0.29,0.53}{#1}}
\newcommand{\DecValTok}[1]{\textcolor[rgb]{0.00,0.00,0.81}{#1}}
\newcommand{\DocumentationTok}[1]{\textcolor[rgb]{0.56,0.35,0.01}{\textbf{\textit{#1}}}}
\newcommand{\ErrorTok}[1]{\textcolor[rgb]{0.64,0.00,0.00}{\textbf{#1}}}
\newcommand{\ExtensionTok}[1]{#1}
\newcommand{\FloatTok}[1]{\textcolor[rgb]{0.00,0.00,0.81}{#1}}
\newcommand{\FunctionTok}[1]{\textcolor[rgb]{0.13,0.29,0.53}{\textbf{#1}}}
\newcommand{\ImportTok}[1]{#1}
\newcommand{\InformationTok}[1]{\textcolor[rgb]{0.56,0.35,0.01}{\textbf{\textit{#1}}}}
\newcommand{\KeywordTok}[1]{\textcolor[rgb]{0.13,0.29,0.53}{\textbf{#1}}}
\newcommand{\NormalTok}[1]{#1}
\newcommand{\OperatorTok}[1]{\textcolor[rgb]{0.81,0.36,0.00}{\textbf{#1}}}
\newcommand{\OtherTok}[1]{\textcolor[rgb]{0.56,0.35,0.01}{#1}}
\newcommand{\PreprocessorTok}[1]{\textcolor[rgb]{0.56,0.35,0.01}{\textit{#1}}}
\newcommand{\RegionMarkerTok}[1]{#1}
\newcommand{\SpecialCharTok}[1]{\textcolor[rgb]{0.81,0.36,0.00}{\textbf{#1}}}
\newcommand{\SpecialStringTok}[1]{\textcolor[rgb]{0.31,0.60,0.02}{#1}}
\newcommand{\StringTok}[1]{\textcolor[rgb]{0.31,0.60,0.02}{#1}}
\newcommand{\VariableTok}[1]{\textcolor[rgb]{0.00,0.00,0.00}{#1}}
\newcommand{\VerbatimStringTok}[1]{\textcolor[rgb]{0.31,0.60,0.02}{#1}}
\newcommand{\WarningTok}[1]{\textcolor[rgb]{0.56,0.35,0.01}{\textbf{\textit{#1}}}}
\usepackage{graphicx}
\makeatletter
\def\maxwidth{\ifdim\Gin@nat@width>\linewidth\linewidth\else\Gin@nat@width\fi}
\def\maxheight{\ifdim\Gin@nat@height>\textheight\textheight\else\Gin@nat@height\fi}
\makeatother
% Scale images if necessary, so that they will not overflow the page
% margins by default, and it is still possible to overwrite the defaults
% using explicit options in \includegraphics[width, height, ...]{}
\setkeys{Gin}{width=\maxwidth,height=\maxheight,keepaspectratio}
% Set default figure placement to htbp
\makeatletter
\def\fps@figure{htbp}
\makeatother
\setlength{\emergencystretch}{3em} % prevent overfull lines
\providecommand{\tightlist}{%
  \setlength{\itemsep}{0pt}\setlength{\parskip}{0pt}}
\setcounter{secnumdepth}{-\maxdimen} % remove section numbering
\ifLuaTeX
  \usepackage{selnolig}  % disable illegal ligatures
\fi
\IfFileExists{bookmark.sty}{\usepackage{bookmark}}{\usepackage{hyperref}}
\IfFileExists{xurl.sty}{\usepackage{xurl}}{} % add URL line breaks if available
\urlstyle{same}
\hypersetup{
  pdftitle={Lab Quiz 1},
  pdfauthor={Fill in your Name},
  hidelinks,
  pdfcreator={LaTeX via pandoc}}

\title{Lab Quiz 1}
\author{Fill in your Name}
\date{Fill in the date}

\begin{document}
\maketitle

\hypertarget{likelihood-methods-for-the-poisson-distribution}{%
\section{Likelihood methods for the Poisson
distribution}\label{likelihood-methods-for-the-poisson-distribution}}

\begin{itemize}
\tightlist
\item
  Put your name in the author section above.
\item
  Write R code in the R chunks provided to answer the questions posed.
\item
  Execute each chunk of code to ensure that your code works properly.
\item
  Sometimes one of your chucks of code will not compile properly, but
  you must hand your document in. In that case, `Comment' out the R code
  that is not working properly using \# as the first character in your
  lines of code.
\item
  Save the Rmd file.
\item
  Knit the Rmd file to pdf.
\item
  Upload the pdf file to the \textbf{Lab Quiz 1 Assignment Activity} in
  the Lab section of Brightspace. * If your file will not knit to pdf,
  then knit to Word and save the Word document as a pdf.
\end{itemize}

\hypertarget{first-generate-1-observation-from-the-poissonlambda5-distribution-and-print-the-value.-2-marks}{%
\subsection{1. First generate 1 observation from the Poisson(lambda=5)
distribution and print the value. {[}2
marks{]}}\label{first-generate-1-observation-from-the-poissonlambda5-distribution-and-print-the-value.-2-marks}}

(Hint: See help for the R function called \emph{rpois}.)

\begin{Shaded}
\begin{Highlighting}[]
\FunctionTok{set.seed}\NormalTok{(}\DecValTok{12345}\NormalTok{)   }\CommentTok{\#use this seed!}
\NormalTok{n }\OtherTok{\textless{}{-}} \DecValTok{1} 
\FunctionTok{rpois}\NormalTok{(n,}\DecValTok{5}\NormalTok{)}
\end{Highlighting}
\end{Shaded}

\begin{verbatim}
## [1] 6
\end{verbatim}

\hypertarget{compute-the-log-likelihood-for-a-vector-sequence-of-lambda-values-from-2-to-7-in-steps-of-.05.-3-marks}{%
\subsection{2. Compute the Log-likelihood for a vector sequence of
lambda values from 2 to 7 in steps of .05. {[}3
marks{]}}\label{compute-the-log-likelihood-for-a-vector-sequence-of-lambda-values-from-2-to-7-in-steps-of-.05.-3-marks}}

(Hint: See help for the R function called \emph{dpois}.)

\begin{Shaded}
\begin{Highlighting}[]
\NormalTok{obs }\OtherTok{\textless{}{-}} \FunctionTok{rpois}\NormalTok{(n,}\DecValTok{5}\NormalTok{)}
\NormalTok{lambda }\OtherTok{\textless{}{-}} \FunctionTok{seq}\NormalTok{(}\DecValTok{2}\NormalTok{,}\DecValTok{7}\NormalTok{,}\FloatTok{0.05}\NormalTok{)}
\NormalTok{dval }\OtherTok{\textless{}{-}} \FunctionTok{dpois}\NormalTok{(obs, lambda, }\AttributeTok{log=}\ConstantTok{TRUE}\NormalTok{)}
\end{Highlighting}
\end{Shaded}

\hypertarget{plot-the-log-likelihood-function-values-from-your-answer-in-2-versus-lambda.-axes-must-be-labelled-and-the-plot-must-have-a-title.-5-marks}{%
\subsection{3. Plot the Log-Likelihood function values from your answer
in 2 versus lambda. Axes must be labelled and the plot must have a
title. {[}5
marks{]}}\label{plot-the-log-likelihood-function-values-from-your-answer-in-2-versus-lambda.-axes-must-be-labelled-and-the-plot-must-have-a-title.-5-marks}}

\begin{Shaded}
\begin{Highlighting}[]
\NormalTok{obs }\OtherTok{\textless{}{-}} \FunctionTok{rpois}\NormalTok{(n,}\DecValTok{5}\NormalTok{)}
\NormalTok{dval }\OtherTok{\textless{}{-}} \FunctionTok{dpois}\NormalTok{(obs, }\FunctionTok{seq}\NormalTok{(}\DecValTok{2}\NormalTok{,}\DecValTok{7}\NormalTok{,}\FloatTok{0.05}\NormalTok{), }\AttributeTok{log=}\ConstantTok{TRUE}\NormalTok{)}
\NormalTok{lambda }\OtherTok{\textless{}{-}} \FunctionTok{seq}\NormalTok{(}\DecValTok{2}\NormalTok{,}\DecValTok{7}\NormalTok{,}\FloatTok{0.05}\NormalTok{)}
\FunctionTok{plot}\NormalTok{(dval}\SpecialCharTok{\textasciitilde{}}\NormalTok{lambda, }\AttributeTok{ylab =} \StringTok{"Log{-}Likelihood"}\NormalTok{, }\AttributeTok{xlab =} \StringTok{"Lambda\textquotesingle{}s"}\NormalTok{, }\AttributeTok{type =} \StringTok{\textquotesingle{}p\textquotesingle{}}\NormalTok{)}
\end{Highlighting}
\end{Shaded}

\includegraphics{Lab1-Assignment-RMD_files/figure-latex/plotLog-1.pdf}

\end{document}
